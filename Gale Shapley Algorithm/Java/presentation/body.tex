
\section{Body}

\subsection{Part 1: Gale-Shapley Algorithm}

\begin{frame}
	\frametitle{Part 1a: Gale-Shapley Algorithm}
	In this algo all the candidates apply simultaneously. \cite{java_tutorial} \pause \\
	On second iteration, those waitlisted apply again to the same program and those rejected apply to the next. \pause \\
	This keeps happening untill all are either waitlisted or out of choices. \pause \\
	Those waitlisted are selected for that course and others don't get a seat. \pause \\
	This algorithm manipulates the preference list of candidates and the merit list in such a way that it automatically takes care of de-reservation. \pause \\
	Each choice is split according to candidates category in the preference list. \pause \\
	Also each program gets its own modified meritlist based on programs category. \pause \\
	Due to these changes we take care of de-reservation in one go of algorithm. 
\end{frame}

\begin{frame}
	\frametitle{Part 1b: Gale-Shapley Algorithm}
	In order to accomodate DS students they are given institute seats before the start of algorithm. \pause \\
	Those who are leftout now apply according to their category in the main algorithm. \pause \\ 
	For foreign candidates, we take care of them right at the end. \pause \\
	Their preference list is also manipulated to take care of de-reserved seat. \pause \\
	After all this, we get an optimal and fair allocation. \pause
\end{frame}

\subsection{Part 2: Merit-Order Algorithm}

\begin{frame}
	\frametitle{Part 2a: Merit-Order Algorithm}
	In this algo we make a common meritlist in which GE, OBC, SC, ST, GE-PD, OBC-PD, SC-PD and ST-PD meritlists all present in increasing order \\
	are appended in the same order. \pause \\
	We go through the list and each candidate(except foreign) gets a chance to apply in its preference list untill he gets an empty seat or his list is exhausted. \pause \\
	Once a candidate gets a seat it is fixed for him and he cannot be replaced. \pause \\
	If a candidate appears multiple times in the CML then he gets multiple chances to go through his preferences unless he already got selected. \pause \\
	After one go of the algorithm we take care of de-reservation i.e vacant seats are added to GE, SC and ST according to rule. \pause \\
	After this we go through the list again as earlier. \pause
\end{frame}

\begin{frame}
	\frametitle{Part 2ab: Merit-Order Algorithm}
	In order to take care of DS category we go through the DS list and apply the same algo for institute seats in place of programs. \pause \\
	After this round all those who were not selected will apply again along with others but this time according to their category. \pause \\
	For foreign candidates, they apply at the end. \pause \\
	We go through their list in the same way as others and there is no need to worry about de-reservation as it has already taken place earlier.\pause \\
	After all this we get an allocation which may not be fair for some cases. \pause
\end{frame}


\subsection{Part 3: Choice Filling Web Application}
\begin{frame}
	\frametitle{Part 3: Choice Filling Web Application}
	\begin{itemize}
		\item We designed a JEE choice filling Web application which lets a student fill his/her choices for JEE counselling \pause \\
		\item We used Python for the coding part - python in Django framework \cite{django_tutorial_1} \pause \\
		\item In this project report, we have tried to elaborate our project work \pause \\
		\item We have also mentioned the references that helped us during the course of the project work \cite{django_tutorial_2} \pause \\
	\end{itemize}
\end{frame}

\begin{frame}
	 The web application is meant to provide the following functionality to the user : \pause \\
	\begin{itemize}
	  \item A user can log into his/her account in the web application \pause \\
	  \item A user can fill his/her choices in the web application - the application takes care of the fact that a particular choice is filled only once \pause \\
	  \item A user can edit his/her existing choices \pause \\
	  \item A user can delete his/her existing choices \pause \\
	  \item A user can check the last year's opening as well as closing ranks for a particular programme \pause \\
	  \item A user can also select an institute and category, and the web application will help the user to decide which branches he/she is likely to get \pause \\
	\end{itemize}
\end{frame}
